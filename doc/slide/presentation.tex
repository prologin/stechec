% ------------------------------------------------------------------------------
% Made by ReeB 4 ACU
%
% Use the Beamer marvelous class
% ------------------------------------------------------------------------------
\documentclass[11pt, handout]{beamer}
\usepackage{graphicx}
\usepackage{wrapfig}
\usepackage[skyblue]{acu-presentation}

\setbeamertemplate{navigation symbols}{}

% ------------------------------------------------------------------------------
% Set here your personnal vars
% ------------------------------------------------------------------------------
\newcommand{\myauthor}{Mathieu \textsc{Betton} \\ St�phane \textsc{Despret} \\ Olivier \textsc{Gournet}}

\newcommand{\mytitle}{\textbf{TowBowlTactics - The revival}}
\newcommand{\myorg}{TBT Team}
\newcommand{\mydate}{\includegraphics[width=5cm]{img/logotbt}\hspace{1cm}}
% ------------------------------------------------------------------------------

\title[TowBowlTactics]{\mytitle}
\author[Mathieu, St�phane, Olivier]{\myauthor}
\institute[RMLL 2006]{\myorg}
\date{\mydate}

% ------------------------------------------------------------------------------
% Some new command
% ------------------------------------------------------------------------------
\newcommand{\coloredblock}[2]{\color{#1}#2\color{black}}
\newcommand{\red}[1]{\coloredblock{red}{#1}}
\newcommand{\black}[1]{\coloredblock{black}{#1}}
\newcommand{\green}[1]{\coloredblock{green}{#1}}
\newcommand{\blue}[1]{\coloredblock{blue}{#1}}
\newcommand{\trans}{}
\newcommand{\lang}[1]{\textbf{\emph{#1}}}
\newcommand{\rb}{\lang{Ruby}}
\newcommand{\img}[2]{\scalebox{#1}[#1]{\includegraphics{#2}}}
\newcommand{\codeimg}[1]{\img{1.0}{generated/#1}}
% ------------------------------------------------------------------------------
 
\begin{document}

\frame{\titlepage}

\logo{}


\frame
{
  \frametitle{Plan de la pr�sentation}
  \tableofcontents[sections={<1-3>}]
}

\section{Le projet TowBowlTactics}

\frame
{
  \begin{block}{Pr�sentation de Blood Bowl}
    \begin{itemize}
      \item Jeu de plateau �dit� par Games Worshop depuis 1986.
      \item Figurines, diff�rentes races.
      \item Allie convivialit�, rapidit�, hasard et strat�gie.
    \end{itemize}
  \end{block}
}

\frame
{
  \begin{block}{Tow Bowl tactics}
    \begin{itemize}
      \item Adaptation vid�o de Blood Bowl.
      \item N�e de la volont�e d'une personne, Pascal Bourut, en 2001.
      \item Programm� en C++/SDL, r�alis� en 1 an.
      \item Communaut�e se formant autour du jeu, mais pas
        suffisamment de programmeur pour le faire vivre.
    \end{itemize}
  \end{block}
}

\frame
{
  \begin{block}{L�galement}
    \begin{itemize}
      \item Projet enti�rement en GPL.
      \item Droits d'adaptation vis-�-vis de GW flou.
    \end{itemize}
  \end{block}
}

\frame
{
  \begin{block}{Reprise fin 2005}
    \begin{itemize}
      \item Reprise amorc�e par un fan: Mathieu.
      \item Soutient technique de Nekeme prod.
      \item Mise en place rapide d'une solide infrastructure:
        \begin{itemize}
          \item site web,
          \item forum,
          \item wiki,
          \item subversion,
          \item mailing-listes.
        \end{itemize}
    \end{itemize}
  \end{block}
}

\frame
{
  \begin{block}{Recrutement et organisation de l'�quipe}
    \begin{itemize}
      \item Pas d'organisation formelle.
      \item Efficacit� certaine.
      \item Importance du document de base d�crivant du point de vue technique le
        projet.
    \end{itemize}
  \end{block}
}

\section{Techniquement}


\frame
{
  \begin{block}{Choix technologique}
    \begin{itemize}
      \item C++/SDL, XML pour la configuration et s'interfacer avec
        d'autres projets.
      \item Doxygen et \LaTeX pour la documentation.
      \item Internationalisation pr�vue.
      \item Le minimum de d�pendances possible.
    \end{itemize}
  \end{block}
}

\frame
{
  \begin{block}{Mod�lisation}
    \begin{itemize}
      \item Syst�me modulaire, client/serveur distinct.
      \item Connexion TCP/IP entre le client et le serveur.
      \item Plusieurs interface possible, console, 2D, \ldots
    \end{itemize}
  \end{block}  
}

\frame
{
  \begin{block}{Multim�dia}
    \begin{itemize}
      \item Les figurines de GW sont pour l'instant utilis�es.
      \item Graphismes � refaire enti�rement.
      \item Petite musique d'ambiance et son...
    \end{itemize}
  \end{block}
  
  \begin{center}
      \includegraphics[width=5cm]{img/towbowl}
  \end{center}
}

\section{Retour d'exp�rience, �volution}

\frame
{
  \begin{block}{Vie du projet}
    \begin{itemize}
      \item Difficult� de motiver des gens ?
      \item Un conseil: faire des backups r�gulier.
    \end{itemize}
  \end{block}
}

\frame
{
  \begin{block}{Futur}
    \begin{itemize}
      \item Release de la premi�re version � la fin de l'�t�.
      \item Pleins d'id�e ! ...
      \item ... Mais trop peu de temps pour les concr�tiser.
    \end{itemize}
  \end{block}
}

\frame
{
  \frametitle{}

  \begin{center}
    {\huge Questions, remarques, informations compl�mentaires ?}
  \end{center}
}

\end{document}
