

%
% Ce fichier a �t� g�n�r� avec gen/make_tex.rtex
% Ne faites pas l'autiste, ne le modifiez pas directement !
%

\subsection{Constantes}\subsubsection{Les codes d'erreur}

\noindent \begin{tabular}{ll}
\textbf{Constante:} & INFINI \\
\textbf{Valeur:} & 30000 \\
\textbf{Description:} & todo \\
\end{tabular} 
\vspace{0.2cm} \\



\noindent \begin{tabular}{ll}
\textbf{Constante:} & HORS\_TERRAIN \\
\textbf{Valeur:} & -1 \\
\textbf{Description:} & todo \\
\end{tabular} 
\vspace{0.2cm} \\



\noindent \begin{tabular}{ll}
\textbf{Constante:} & PAS\_DE\_MAISON \\
\textbf{Valeur:} & -2 \\
\textbf{Description:} & todo \\
\end{tabular} 
\vspace{0.2cm} \\



\noindent \begin{tabular}{ll}
\textbf{Constante:} & PAS\_DE\_MONUMENT \\
\textbf{Valeur:} & -3 \\
\textbf{Description:} & todo \\
\end{tabular} 
\vspace{0.2cm} \\



\noindent \begin{tabular}{ll}
\textbf{Constante:} & FINANCES\_DEPASSEES \\
\textbf{Valeur:} & -4 \\
\textbf{Description:} & todo \\
\end{tabular} 
\vspace{0.2cm} \\



\noindent \begin{tabular}{ll}
\textbf{Constante:} & BLOCAGE \\
\textbf{Valeur:} & -5 \\
\textbf{Description:} & todo \\
\end{tabular} 
\vspace{0.2cm} \\



\noindent \begin{tabular}{ll}
\textbf{Constante:} & TROP\_DE\_MAISONS \\
\textbf{Valeur:} & -6 \\
\textbf{Description:} & todo \\
\end{tabular} 
\vspace{0.2cm} \\



\noindent \begin{tabular}{ll}
\textbf{Constante:} & JOUEUR\_INCORRECT \\
\textbf{Valeur:} & -7 \\
\textbf{Description:} & todo \\
\end{tabular} 
\vspace{0.2cm} \\



\noindent \begin{tabular}{ll}
\textbf{Constante:} & NON\_CONNEXE \\
\textbf{Valeur:} & -8 \\
\textbf{Description:} & todo \\
\end{tabular} 
\vspace{0.2cm} \\



\noindent \begin{tabular}{ll}
\textbf{Constante:} & CASE\_OCCUPEE \\
\textbf{Valeur:} & -9 \\
\textbf{Description:} & todo \\
\end{tabular} 
\vspace{0.2cm} \\



\noindent \begin{tabular}{ll}
\textbf{Constante:} & ACTION\_INTERDITE \\
\textbf{Valeur:} & -10 \\
\textbf{Description:} & todo \\
\end{tabular} 
\vspace{0.2cm} \\



\noindent \begin{tabular}{ll}
\textbf{Constante:} & TROP\_LOIN \\
\textbf{Valeur:} & -11 \\
\textbf{Description:} & todo \\
\end{tabular} 
\vspace{0.2cm} \\



\noindent \begin{tabular}{ll}
\textbf{Constante:} & SUCCES \\
\textbf{Valeur:} & 0 \\
\textbf{Description:} & todo \\
\end{tabular} 
\vspace{0.2cm} \\

\subsubsection{Les constantes de possession}

\noindent \begin{tabular}{ll}
\textbf{Constante:} & MAIRIE \\
\textbf{Valeur:} & 3 \\
\textbf{Description:} & todo \\
\end{tabular} 
\vspace{0.2cm} \\

\subsubsection{Les constantes de terrain}

\noindent \begin{tabular}{ll}
\textbf{Constante:} & VIDE \\
\textbf{Valeur:} & 0 \\
\textbf{Description:} & Case de terrain vide \\
\end{tabular} 
\vspace{0.2cm} \\



\noindent \begin{tabular}{ll}
\textbf{Constante:} & MAISON \\
\textbf{Valeur:} & 1 \\
\textbf{Description:} & Case de terrain qui contient une maison \\
\end{tabular} 
\vspace{0.2cm} \\



\noindent \begin{tabular}{ll}
\textbf{Constante:} & RESERVATION \\
\textbf{Valeur:} & 2 \\
\textbf{Description:} & Case de terrain r�serv�e \\
\end{tabular} 
\vspace{0.2cm} \\



\noindent \begin{tabular}{ll}
\textbf{Constante:} & MONUMENT \\
\textbf{Valeur:} & 3 \\
\textbf{Description:} & Case de terrain qui contient un monument \\
\end{tabular} 
\vspace{0.2cm} \\



\noindent \begin{tabular}{ll}
\textbf{Constante:} & ROUTE \\
\textbf{Valeur:} & 4 \\
\textbf{Description:} & Case de terrain qui contient une route \\
\end{tabular} 
\vspace{0.2cm} \\

\subsubsection{Bornes, tailles}

\noindent \begin{tabular}{ll}
\textbf{Constante:} & MAX\_MONUMENTS \\
\textbf{Valeur:} & 13 \\
\textbf{Description:} & todo \\
\end{tabular} 
\vspace{0.2cm} \\



\noindent \begin{tabular}{ll}
\textbf{Constante:} & TAILLE\_CARTE \\
\textbf{Valeur:} & 100 \\
\textbf{Description:} & todo \\
\end{tabular} 
\vspace{0.2cm} \\



\subsection{Fonctions d'information}
Toutes les fonctions peuvent renvoyer les constantes
\textbf{BAD\_ARGUMENT} quand au moins un des
arguments est incorrect (m�me si cela n'est pas pr�cis� pour chaque
fonction). Par exemple, cela se produit si vous appelez
la fonction \texttt{type\_case} avec \texttt{x=13} et \texttt{y=5142}.

\begin{minipage}{\linewidth}
\functitle{type\_case}

\begin{lst-c++}
int type_case(int x, int y)
\end{lst-c++}

\noindent
\begin{tabular}[t]{@{\extracolsep{0pt}}>{\bfseries}lp{10cm}}
Description~: & Renvoie le type de la case \\


Parametres~: &
\begin{tabular}[t]{@{\extracolsep{0pt}}ll}
    
    
      
        \textsl{x}~: & todo \\
      
    
      
        \textsl{y}~: & todo \\
      
    
  \end{tabular} \\






\end{tabular} \\[0.3cm]
\end{minipage}


\begin{minipage}{\linewidth}
\functitle{valeur\_case}

\begin{lst-c++}
int valeur_case(int x, int y)
\end{lst-c++}

\noindent
\begin{tabular}[t]{@{\extracolsep{0pt}}>{\bfseries}lp{10cm}}
Description~: & Renvoie la valeur de la case \\


Parametres~: &
\begin{tabular}[t]{@{\extracolsep{0pt}}ll}
    
    
      
        \textsl{x}~: & todo \\
      
    
      
        \textsl{y}~: & todo \\
      
    
  \end{tabular} \\






\end{tabular} \\[0.3cm]
\end{minipage}


\begin{minipage}{\linewidth}
\functitle{appartenance}

\begin{lst-c++}
int appartenance(int x, int y)
\end{lst-c++}

\noindent
\begin{tabular}[t]{@{\extracolsep{0pt}}>{\bfseries}lp{10cm}}
Description~: & todo \\


Parametres~: &
\begin{tabular}[t]{@{\extracolsep{0pt}}ll}
    
    
      
        \textsl{x}~: & todo \\
      
    
      
        \textsl{y}~: & todo \\
      
    
  \end{tabular} \\






\end{tabular} \\[0.3cm]
\end{minipage}


\begin{minipage}{\linewidth}
\functitle{type\_monument}

\begin{lst-c++}
int type_monument(int x, int y)
\end{lst-c++}

\noindent
\begin{tabular}[t]{@{\extracolsep{0pt}}>{\bfseries}lp{10cm}}
Description~: & todo \\


Parametres~: &
\begin{tabular}[t]{@{\extracolsep{0pt}}ll}
    
    
      
        \textsl{x}~: & todo \\
      
    
      
        \textsl{y}~: & todo \\
      
    
  \end{tabular} \\






\end{tabular} \\[0.3cm]
\end{minipage}


\begin{minipage}{\linewidth}
\functitle{portee\_monument}

\begin{lst-c++}
int portee_monument(int num_monument)
\end{lst-c++}

\noindent
\begin{tabular}[t]{@{\extracolsep{0pt}}>{\bfseries}lp{10cm}}
Description~: & todo \\


Parametres~: &
\begin{tabular}[t]{@{\extracolsep{0pt}}ll}
    
    
      
        \textsl{num\_monument}~: & todo \\
      
    
  \end{tabular} \\






\end{tabular} \\[0.3cm]
\end{minipage}


\begin{minipage}{\linewidth}
\functitle{prestige\_monument}

\begin{lst-c++}
int prestige_monument(int num_monument)
\end{lst-c++}

\noindent
\begin{tabular}[t]{@{\extracolsep{0pt}}>{\bfseries}lp{10cm}}
Description~: & todo \\


Parametres~: &
\begin{tabular}[t]{@{\extracolsep{0pt}}ll}
    
    
      
        \textsl{num\_monument}~: & todo \\
      
    
  \end{tabular} \\






\end{tabular} \\[0.3cm]
\end{minipage}


\begin{minipage}{\linewidth}
\functitle{numero\_tour}

\begin{lst-c++}
int numero_tour()
\end{lst-c++}

\noindent
\begin{tabular}[t]{@{\extracolsep{0pt}}>{\bfseries}lp{10cm}}
Description~: & Renvoie le numero du tour \\







\end{tabular} \\[0.3cm]
\end{minipage}


\begin{minipage}{\linewidth}
\functitle{commence}

\begin{lst-c++}
int commence()
\end{lst-c++}

\noindent
\begin{tabular}[t]{@{\extracolsep{0pt}}>{\bfseries}lp{10cm}}
Description~: & Numero du joueur qui commence \\







\end{tabular} \\[0.3cm]
\end{minipage}


\begin{minipage}{\linewidth}
\functitle{montant\_encheres}

\begin{lst-c++}
int montant_encheres(int num_joueur)
\end{lst-c++}

\noindent
\begin{tabular}[t]{@{\extracolsep{0pt}}>{\bfseries}lp{10cm}}
Description~: & todo \\


Parametres~: &
\begin{tabular}[t]{@{\extracolsep{0pt}}ll}
    
    
      
        \textsl{num\_joueur}~: & todo \\
      
    
  \end{tabular} \\






\end{tabular} \\[0.3cm]
\end{minipage}


\begin{minipage}{\linewidth}
\functitle{score}

\begin{lst-c++}
int score(int num_joueur)
\end{lst-c++}

\noindent
\begin{tabular}[t]{@{\extracolsep{0pt}}>{\bfseries}lp{10cm}}
Description~: & todo \\


Parametres~: &
\begin{tabular}[t]{@{\extracolsep{0pt}}ll}
    
    
      
        \textsl{num\_joueur}~: & id du joueur \\
      
    
  \end{tabular} \\






\end{tabular} \\[0.3cm]
\end{minipage}


\begin{minipage}{\linewidth}
\functitle{finances}

\begin{lst-c++}
int finances(int num_joueur)
\end{lst-c++}

\noindent
\begin{tabular}[t]{@{\extracolsep{0pt}}>{\bfseries}lp{10cm}}
Description~: & todo \\


Parametres~: &
\begin{tabular}[t]{@{\extracolsep{0pt}}ll}
    
    
      
        \textsl{num\_joueur}~: & id du joueur \\
      
    
  \end{tabular} \\






\end{tabular} \\[0.3cm]
\end{minipage}


\begin{minipage}{\linewidth}
\functitle{monument\_en\_cours}

\begin{lst-c++}
int monument_en_cours()
\end{lst-c++}

\noindent
\begin{tabular}[t]{@{\extracolsep{0pt}}>{\bfseries}lp{10cm}}
Description~: & todo \\







\end{tabular} \\[0.3cm]
\end{minipage}


\begin{minipage}{\linewidth}
\functitle{distance}

\begin{lst-c++}
int distance(int x1, int y1, int x2, int y2)
\end{lst-c++}

\noindent
\begin{tabular}[t]{@{\extracolsep{0pt}}>{\bfseries}lp{10cm}}
Description~: & todo \\


Parametres~: &
\begin{tabular}[t]{@{\extracolsep{0pt}}ll}
    
    
      
        \textsl{x1}~: & la colonne du premier point \\
      
    
      
        \textsl{y1}~: & la ligne du premier point \\
      
    
      
        \textsl{x2}~: & la colonne du second point \\
      
    
      
        \textsl{y2}~: & la ligne du second point \\
      
    
  \end{tabular} \\






\end{tabular} \\[0.3cm]
\end{minipage}


\begin{minipage}{\linewidth}
\functitle{route\_possible}

\begin{lst-c++}
int route_possible(int x, int y)
\end{lst-c++}

\noindent
\begin{tabular}[t]{@{\extracolsep{0pt}}>{\bfseries}lp{10cm}}
Description~: & todo \\


Parametres~: &
\begin{tabular}[t]{@{\extracolsep{0pt}}ll}
    
    
      
        \textsl{x}~: & la colonne \\
      
    
      
        \textsl{y}~: & la ligne \\
      
    
  \end{tabular} \\






\end{tabular} \\[0.3cm]
\end{minipage}


\begin{minipage}{\linewidth}
\functitle{construction\_possible}

\begin{lst-c++}
int construction_possible(int x, int y)
\end{lst-c++}

\noindent
\begin{tabular}[t]{@{\extracolsep{0pt}}>{\bfseries}lp{10cm}}
Description~: & todo \\


Parametres~: &
\begin{tabular}[t]{@{\extracolsep{0pt}}ll}
    
    
      
        \textsl{x}~: & la colonne \\
      
    
      
        \textsl{y}~: & la ligne \\
      
    
  \end{tabular} \\






\end{tabular} \\[0.3cm]
\end{minipage}


\begin{minipage}{\linewidth}
\functitle{cout\_achat\_maison}

\begin{lst-c++}
int cout_achat_maison()
\end{lst-c++}

\noindent
\begin{tabular}[t]{@{\extracolsep{0pt}}>{\bfseries}lp{10cm}}
Description~: & todo \\







\end{tabular} \\[0.3cm]
\end{minipage}


\begin{minipage}{\linewidth}
\functitle{cout\_achat\_route}

\begin{lst-c++}
int cout_achat_route()
\end{lst-c++}

\noindent
\begin{tabular}[t]{@{\extracolsep{0pt}}>{\bfseries}lp{10cm}}
Description~: & todo \\







\end{tabular} \\[0.3cm]
\end{minipage}


\begin{minipage}{\linewidth}
\functitle{mon\_tour}

\begin{lst-c++}
int mon_tour()
\end{lst-c++}

\noindent
\begin{tabular}[t]{@{\extracolsep{0pt}}>{\bfseries}lp{10cm}}
Description~: & todo \\







\end{tabular} \\[0.3cm]
\end{minipage}

\subsection{Fonctions d'action}
Toutes les fonctions peuvent renvoyer les constantes
\textbf{BAD\_ARGUMENT} quand au moins un des
arguments est incorrect (m�me si cela n'est pas pr�cis� pour chaque
fonction), ou \textbf{TOO\_MUCH\_ORDERS} quand vous avez envoy� trop 
d'ordres � l'un de vos hamsters. TODO

\begin{minipage}{\linewidth}
\functitle{construire\_route}

\begin{lst-c++}
int construire_route(int x, int y)
\end{lst-c++}

\noindent
\begin{tabular}[t]{@{\extracolsep{0pt}}>{\bfseries}lp{10cm}}
Description~: & todo \\


Parametres~: &
\begin{tabular}[t]{@{\extracolsep{0pt}}ll}
    
    
      
        \textsl{x}~: & la colonne \\
      
    
      
        \textsl{y}~: & la ligne \\
      
    
  \end{tabular} \\






\end{tabular} \\[0.3cm]
\end{minipage}


\begin{minipage}{\linewidth}
\functitle{construire\_maison}

\begin{lst-c++}
int construire_maison(int x, int y)
\end{lst-c++}

\noindent
\begin{tabular}[t]{@{\extracolsep{0pt}}>{\bfseries}lp{10cm}}
Description~: & todo \\


Parametres~: &
\begin{tabular}[t]{@{\extracolsep{0pt}}ll}
    
    
      
        \textsl{x}~: & la colonne \\
      
    
      
        \textsl{y}~: & la ligne \\
      
    
  \end{tabular} \\






\end{tabular} \\[0.3cm]
\end{minipage}


\begin{minipage}{\linewidth}
\functitle{reserver\_case}

\begin{lst-c++}
int reserver_case(int x, int y)
\end{lst-c++}

\noindent
\begin{tabular}[t]{@{\extracolsep{0pt}}>{\bfseries}lp{10cm}}
Description~: & todo \\


Parametres~: &
\begin{tabular}[t]{@{\extracolsep{0pt}}ll}
    
    
      
        \textsl{x}~: & la colonne \\
      
    
      
        \textsl{y}~: & la ligne \\
      
    
  \end{tabular} \\






\end{tabular} \\[0.3cm]
\end{minipage}


\begin{minipage}{\linewidth}
\functitle{detruire\_maison}

\begin{lst-c++}
int detruire_maison(int x, int y)
\end{lst-c++}

\noindent
\begin{tabular}[t]{@{\extracolsep{0pt}}>{\bfseries}lp{10cm}}
Description~: & todo \\


Parametres~: &
\begin{tabular}[t]{@{\extracolsep{0pt}}ll}
    
    
      
        \textsl{x}~: & la colonne \\
      
    
      
        \textsl{y}~: & la ligne \\
      
    
  \end{tabular} \\






\end{tabular} \\[0.3cm]
\end{minipage}


\begin{minipage}{\linewidth}
\functitle{vendre\_maison}

\begin{lst-c++}
int vendre_maison(int x, int y)
\end{lst-c++}

\noindent
\begin{tabular}[t]{@{\extracolsep{0pt}}>{\bfseries}lp{10cm}}
Description~: & todo \\


Parametres~: &
\begin{tabular}[t]{@{\extracolsep{0pt}}ll}
    
    
      
        \textsl{x}~: & la colonne \\
      
    
      
        \textsl{y}~: & la ligne \\
      
    
  \end{tabular} \\






\end{tabular} \\[0.3cm]
\end{minipage}


\begin{minipage}{\linewidth}
\functitle{encherir}

\begin{lst-c++}
int encherir(int montant)
\end{lst-c++}

\noindent
\begin{tabular}[t]{@{\extracolsep{0pt}}>{\bfseries}lp{10cm}}
Description~: & todo \\


Parametres~: &
\begin{tabular}[t]{@{\extracolsep{0pt}}ll}
    
    
      
        \textsl{montant}~: & Le montant de votre ench�re \\
      
    
  \end{tabular} \\






\end{tabular} \\[0.3cm]
\end{minipage}


\begin{minipage}{\linewidth}
\functitle{construire\_monument}

\begin{lst-c++}
int construire_monument(int x, int y)
\end{lst-c++}

\noindent
\begin{tabular}[t]{@{\extracolsep{0pt}}>{\bfseries}lp{10cm}}
Description~: & todo \\


Parametres~: &
\begin{tabular}[t]{@{\extracolsep{0pt}}ll}
    
    
      
        \textsl{x}~: & la colonne \\
      
    
      
        \textsl{y}~: & la ligne \\
      
    
  \end{tabular} \\






\end{tabular} \\[0.3cm]
\end{minipage}



\end{document}
