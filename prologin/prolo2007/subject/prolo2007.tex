%LaTeX Document

\documentclass[a4paper,twoside,12pt]{article}

\usepackage{a4}
\usepackage{geometry}
\usepackage{multicol}
\usepackage{fancyhdr}
\usepackage{subfigure}
\usepackage{verbatim}

\usepackage{graphicx}
\usepackage[latin1]{inputenc}
\usepackage[T1]{fontenc}
\usepackage[french]{babel}
\usepackage{subject}
\usepackage{eurosym}

%\NoAutoSpaceBeforeFDP

\makeindex

\begin{document}

\sloppy

\lhead[\textsl{Prologin 2007}]{\nouppercase \leftmark}
\rhead[Sujet de finale]{}

% Couverture =========================================================
\begin{titlepage}
\begin{center}
\includegraphics[width=\linewidth]{logo_prologin2007} \\
\vspace{4cm}
\Huge
\textbf{Il \'etait une fois la vie}\\
\vspace{2cm}
\normalsize
Sujet de la finale du Concours National d'Informatique\\
28 au 30 avril 2007\\
%\vspace{2cm}
%\large(Derni�re r�vision~: \date)\\
\end{center}
\end{titlepage}


% Sommaire ===========================================================
\cleardoublepage
\tableofcontents

% Corps ==============================================================
\cleardoublepage
\setcounter{page}{1}
\pagestyle{fancy}
\parskip=6pt plus 3pt

\section{Il \'etait une fois\ldots} %-----------------------------------

\subsection{\ldots{} la vie}

TODO: intro

\subsection{Super-globule est arriv\'e}

Le  centre  de contr\^ole  du  corps humain  \`a  le  plaisir de  vous
annoncer l'ouverture du concours hebdomadaire Super-globule. A l'issue
d'\'epreuves plus difficiles les unes que les autres, le globule blanc
le  plus d\'evou\'e,  performant,  et acharn\'e  au  travail se  verra
decern\'e le  titre de Super-globule,  et pourra ainsi  pr\'etendre au
tr\`es convoit\'e  poste de  chef de la  s\'ecurit\'e du  corps humain
pour une dur\'ee d'une semaine.

Le  concours dispose  de r\`egles  simples  : pendant  36 heures,  des
\'equipes  de globules  blancs devront  d\'emontrer leur  talents dans
diff\'erents endroits du corps humain en prot\'egeant les cellules des
attaques virales et bact\'eriologiques.  Plus une \'equipe d\'efend de
cellules, plus elle  gagne de points. Mais chaque  cellule perdue fera
reculer de mani\`ere catastrophique le score de l'\'equipe cens\'ee la
prot\'eger. \`A la fin de  ces \'epreuves, c'est le chef de l'\'equipe
qui aura \'et\'e  la plus performante et qui  aura prot\'eg\'e le plus
de cellules qui recevra le titre de Super-globule.

Mais  ce n'est  pas simple  de devenir  Super-globule,  n'importe quel
globule  quelconque  n'a  pas  toutes les  qualit\'es  requises.  Tout
d'abord, il  vous faudra avoir  l'esprit strat\'egique, et  former une
\'equipe. La  r\'ecompense pour chaque membre  de l'\'equipe gagnante,
\'etant une journ\'ee sabbatique  en compagnie de globules rouges tous
plus  charmants  les  uns  que   les  autres,  attire  elle  aussi  la
convoitise.  Vous  aurez donc  l'embarras  du  choix  pour former  vos
\'equipes. Parmi  les caract\'eristiques  de chaque globule  blanc, on
compte : la vitesse, la robustesse et la force d'attaque.

La deuxi\`eme qualit\'e dont  doit \^etre pourvu Super-globule est une
technique de  combat incomparable. Des virus  et bact\'eries attaquent
sans cesse le  corps humain. Chaque \'equipe de  globules blancs devra
prot\'eger un  maximum de cellules  de ces attaques. Un  globule blanc
touche une cellule pour lui indiquer qu'il la prot\`ege.
% Transition
Chaque  virus ou  bact\'erie neutralis\'ee  par un  quelconque globule
blanc rapportera des munitions \`a l'\'equipe qui prot\`ege la cellule
qui \'etait vis\'ee par l'attaque.

Enfin, il va de soi  que Super-globule doit \^etre dot\'e d'un courage
sans limites  et doit  \^etre pr\^et \`a  se sacrifier  pour n'importe
laquelle des cellules du corps humain.

\subsection{10 000 milliards de cellules, et moi et moi et moi}

Le corps humain  est un organisme vaste o\`u  chaque partie remplit un
r\^ole   bien   particulier  mais   indispensable.   Le  concours   de
Super-globule  prendra place dans  diff\'erents organes,  ayant chacun
leurs particularit\'es.

Les  muscles  sont  des   tissus  form\'es  de  cellules  contractiles
appel\'ees \emph{myocyte}.  Les muscles sont  moder\'ements sujets aux
attaques et leur concentration en cellules est assez \'elev\'ee.

Le coeur  est l'organe creux  musculaire qui assure la  circulation du
sang dans  tout le reste du  corps. Il est donc  tr\`es difficile pour
les  globules  blancs  d'assurer  leur  travail,  car  ce  muscle  est
syst\'ematiquement  en  contraction   (cela  provoque  des  mouvements
ind\'esirables).

Les  poumons sont  tapiss\'es de  cellules. Et  ils sont  aussi tr\`es
sujets aux aggressions exterieures, alors restez sur vos gardes !

L'intestin gr\^ele  est l'organe le plus cons\'equent  du corps humain
(entre 5  et 7 m\`etres), o\`u  la concentration en  cellules de types
vari\'es est tr\'es  \'elev\'ee. Gare aux aliments qui  y circulent et
qui pourraient vous d\'evier en cas de collision.

Le  cerveau est  l'organe qui  contr\^ole le  syst\`eme nerveux.  On y
compte aussi  beaucoup de cellules.  La perte malencontreuse  d'une de
ses cellules pourrait bien provoquer des messages nerveux violents, et
totalement destabiliser les globules blancs protecteurs.

\subsection{Le quartier g\'en\'eral des globules}

Votre quartier  g\'en\'eral (Q.G.)  est  le lieu de d\'epart  pour vos
globules blancs. Vous aurez un  nombre fixe de globules \`a prendre en
main et  \`a diriger. Ce nombre  ne pourra pas  augmenter, mais pourra
facilement baisser (infections, attaques).

Avant de commencer son difficile travail, chaque globule doit subir un
entra\^inement  express.  Etant  donn\'e  le peu  de  temps dont  vous
disposez  lorsque vous  avez besoin  d'int\'egrer un  nouveau globule,
cette phase se r\'esume \`a une sorte de dopage (oui, c'est autoris\'e
!)   \`a l'oxyg\`ene,  aux prote\"ines  et aux  lipides.  Lorsque vous
engagez un nouvel  \'equipier, vous aurez donc une  quantit\'e fixe de
points  \`a disposiiton  pour lui  acheter et  lui offrir  de  quoi se
doper.  Il vous  faudra r\'epartir  ces points  de la  meilleure facon
possible, libre \`a vous ensuite de lui apporter plus d'oxyg\`ene pour
qu'il se d\'eplace plus vite, plus de prote\"ines pour qu'il soit plus
r\'esistant aux  attaques et enfin  plus de lipides pour  qu'il puisse
attaquer. En effet, on ne vous l'a pas dit : les lipides sont id\'eaux
pour noyer les  virus et bact\'eries dans le  gras. Utilis\'es sur les
globules concurrents,  ils les feront patauger sur  place de mani\`ere
ridicule en leur faisant perdre tout efficacit\'e !

Mais prenez  soin de votre  Q.G. Une \'equipe adverse  pourrait tr\`es
bien  l'attaquer  et ainsi  emp\'echer  temporairement votre  \'equipe
d'engager de nouveaux globules !

\subsection{Donner sa vie pour les autres}

On vous l'a d\'ej\`a dit, mais  on n'h\'esitera pas \`a vous le redire
:  votre  mission est  de  prot\'eger les  cellules  du  corps, et  ce
quelqu'en soit le prix ! Car  si les virus attaquent les cellules, ils
peuvent aussi attaquer les globules blancs et les tuer.

% Transition bizarre: les virus tuent globule --> proteger cellule

Votre   mission  est   donc  simple   :  prot\'eger   un   maximum  de
cellules. Pour que  cela vous rapporte des points,  vous devez d'abord
choisir les cellules que vous allez  d\'efendre. Il n'y a rien de plus
simple : il  vous suffit de toucher une cellule  pour lui indiquer que
vous \^etes  son garde  du corps\ldots{} Bien  entendu, si  un globule
d'une \'equipe adverse touche \`a son tour cette m\^eme cellule, alors
il  lui indique  que  c'est  d\'esormais cette  \'equipe  l\`a qui  la
prot\`ege.

Par la  suite, il y a  deux cas de figure.  Le pire :  une cellule que
vous  d\'efendez est  infect\'ee est  meurt. Dans  ce cas  l\`a, votre
score   sera   rabaiss\'e    d'un   nombre   non   n'\'egligeable   de
points. L'autre cas est bien s\^ur que vos cellules survivent. Dans ce
dernier, chaque fois que vous  \'eliminerez un virus ou une bact\'erie
sur le  point d'attaquer  une des cellules  que vous  prot\'egez, vous
gagnerez des points.


\subsection{Allo oui, j'\'ecoute !}

La  t\^ache vous  semble ardue  et pleine  de strat\'egies.  Pour vous
aider \`a coordonner vos unit\'es, vous aurez le besoin de communiquer
entre  unit\'es.  Les  buts de  ces communications  sont  multiples et
vari\'es:

\begin{itemize}
\item Coordonner  les actions  des unit\'es afin  de se  r\'epartir le
travail (en gain de cellules ou bien m\^eme en attaque).
\item Se communiquer les \'etats de sant\'e.
\item Et bien d'autres possibilit\'es.
\end{itemize}

Pour ce  faire, vous disposerez  d'une fonction de base,  qui enverras
aux  coll\`egues des messages  sous la  forme d'entiers,  ayant chacun
leur signification.  Nous vous proposons un d\'ebut de liste pour vous
donner des  id\'ees de messages, cependant vous  pourrez inventer tous
les messages que vous d\'esirerez.

% A finir

\subsection{Les attaques virales et bact\'eriologiques}

Comme vous l'avez sans doute  compris, il existe deux types d'attaques
: les bact\'eries, qui sont assez faibles et qui se d\'eplacent tr\`es
vite,  et les  virus,  qui sont  plus  violents. Chacun  de ces  corps
d\'ecide  lors de  son arriv\'ee  dans  le corps  d'une cellule  cible
al\'eatoire, et va ainsi se diriger et aller attaquer cette cellule.

Lorsqu'une cellule  est infect\'ee par  une bact\'erie, elle  meurt au
bout  d'un  certain temps,  sauf  si  un  globule vient  d\'eloger  la
bact\'erie. Si par contre la cellule est touch\'ee par un virus, alors
cette derni\`ere meurt bien plus  rapidement et se transforme alors en
un deuxi\`eme virus !

La fr\'equence  d'apparition de ces ennemis est  r\'eguli\`ere et vaut
FREQVIRUS.  Le nombre de  virus qui appara\^it lors de ces
attaques, est d\'ependant  du nombre de cellules encore  en jeu, ainsi
que de la difficult\'e de la charge.

Autant les  bact\'eries sont inoffensives envers  les globules blancs,
autant  les virus vous  sont mortels  (dans les  m\^eme circonstances,
vous  vous transformez alors  en virus  vous m\^eme)\ldots{}  Gare \`a
vous.

Vous  voyez  donc que  cela  ne rigole  pas  :  vous \^etes  oblig\'es
d'utiliser    la    force    contre    ces    menaces    virales    et
bact\'eriologiques. Pour  cela, vous  avez deux possibilit\'es  : vous
jeter sur  un corps hostile et  l'\'etouffer avec des  lipides ou bien
lui jeter \`a distance des lipides.

\subsection{Super-globule\ldots{} par tous les moyens !}

Tout  les  coups  sont  permis  dans  la bataille  pour  le  titre  de
Super-globule.  Y  compris mettre  les  b\^atons  dans  les roues  des
globules blancs des \'equipes adverses !

Votre objectif  principal est cependant bien plus  important : inutile
d'esp\'erer  avoir  des  points  en  ne  vous  affrontant  qu'\`a  vos
adversaires et  pas au virus  et bact\'eries. Pour cela,  les r\`egles
sont strictes : attaquer un globule blanc ne rapporte aucun points.

Vous avez  la possibilit\'e de ralentir les  \'equipes concurrentes en
les  attaquant avec  des lipides  :  chaque globule  touch\'e par  ces
mati\`eres  se verra  fortement ralenti  et fera  m\^eme  du sur-place
pendant quelques instants.

Autre  coup bas  :  attaquer le  Q.G.   d'une autre  \'equipe afin  de
l'emp\^echer d'augmenter des effectifs en globules blancs.

\subsection{Calcul des points d'une \'equipe}

Le calcul des points est tr\`es simple.

La seule chose qui vous rapporte des points, c'est de tuer un virus ou
une bact\'erie.  Cependant attention ! Les points  sont revers\'es \`a
l'\'equipe   qui  prot\'ege   la   cellule  cibl\'ee   par  le   corps
hostile. Comprenez par  l\`a que si un virus a  pour cible une cellule
et que  vous n'avez pas touch\'e  cette cellule pour  lui indiquer que
vous la prot\'egez, alors aucun point ne vous sera donn\'e.

La  deuxi\`eme  chose  qui  affecte  votre score  est  la  mort  d'une
cellule. Si vous avez touch\'e  une cellule pour lui indiquer que vous
la defendez  et que  cette derni\`ere est  amen\'ee \`a  mourir, alors
vous perdrez des points.

\section{Les actions possibles} %-------------------------------------

TODO



\newpage
\section{Outils informatiques} %--------------------------------------




%
% Ce fichier a �t� g�n�r� avec gen/make_tex.rtex
% Ne faites pas l'autiste, ne le modifiez pas directement !
%

\subsection{Constantes}\subsubsection{Les codes d'erreur}

\noindent \begin{tabular}{ll}
\textbf{Constante:} & INFINI \\
\textbf{Valeur:} & 30000 \\
\textbf{Description:} & todo \\
\end{tabular} 
\vspace{0.2cm} \\



\noindent \begin{tabular}{ll}
\textbf{Constante:} & HORS\_TERRAIN \\
\textbf{Valeur:} & -1 \\
\textbf{Description:} & todo \\
\end{tabular} 
\vspace{0.2cm} \\



\noindent \begin{tabular}{ll}
\textbf{Constante:} & PAS\_DE\_MAISON \\
\textbf{Valeur:} & -2 \\
\textbf{Description:} & todo \\
\end{tabular} 
\vspace{0.2cm} \\



\noindent \begin{tabular}{ll}
\textbf{Constante:} & PAS\_DE\_MONUMENT \\
\textbf{Valeur:} & -3 \\
\textbf{Description:} & todo \\
\end{tabular} 
\vspace{0.2cm} \\



\noindent \begin{tabular}{ll}
\textbf{Constante:} & FINANCES\_DEPASSEES \\
\textbf{Valeur:} & -4 \\
\textbf{Description:} & todo \\
\end{tabular} 
\vspace{0.2cm} \\



\noindent \begin{tabular}{ll}
\textbf{Constante:} & BLOCAGE \\
\textbf{Valeur:} & -5 \\
\textbf{Description:} & todo \\
\end{tabular} 
\vspace{0.2cm} \\



\noindent \begin{tabular}{ll}
\textbf{Constante:} & TROP\_DE\_MAISONS \\
\textbf{Valeur:} & -6 \\
\textbf{Description:} & todo \\
\end{tabular} 
\vspace{0.2cm} \\



\noindent \begin{tabular}{ll}
\textbf{Constante:} & JOUEUR\_INCORRECT \\
\textbf{Valeur:} & -7 \\
\textbf{Description:} & todo \\
\end{tabular} 
\vspace{0.2cm} \\



\noindent \begin{tabular}{ll}
\textbf{Constante:} & NON\_CONNEXE \\
\textbf{Valeur:} & -8 \\
\textbf{Description:} & todo \\
\end{tabular} 
\vspace{0.2cm} \\



\noindent \begin{tabular}{ll}
\textbf{Constante:} & CASE\_OCCUPEE \\
\textbf{Valeur:} & -9 \\
\textbf{Description:} & todo \\
\end{tabular} 
\vspace{0.2cm} \\



\noindent \begin{tabular}{ll}
\textbf{Constante:} & ACTION\_INTERDITE \\
\textbf{Valeur:} & -10 \\
\textbf{Description:} & todo \\
\end{tabular} 
\vspace{0.2cm} \\



\noindent \begin{tabular}{ll}
\textbf{Constante:} & TROP\_LOIN \\
\textbf{Valeur:} & -11 \\
\textbf{Description:} & todo \\
\end{tabular} 
\vspace{0.2cm} \\



\noindent \begin{tabular}{ll}
\textbf{Constante:} & SUCCES \\
\textbf{Valeur:} & 0 \\
\textbf{Description:} & todo \\
\end{tabular} 
\vspace{0.2cm} \\

\subsubsection{Les constantes de possession}

\noindent \begin{tabular}{ll}
\textbf{Constante:} & MAIRIE \\
\textbf{Valeur:} & 3 \\
\textbf{Description:} & todo \\
\end{tabular} 
\vspace{0.2cm} \\

\subsubsection{Les constantes de terrain}

\noindent \begin{tabular}{ll}
\textbf{Constante:} & VIDE \\
\textbf{Valeur:} & 0 \\
\textbf{Description:} & Case de terrain vide \\
\end{tabular} 
\vspace{0.2cm} \\



\noindent \begin{tabular}{ll}
\textbf{Constante:} & MAISON \\
\textbf{Valeur:} & 1 \\
\textbf{Description:} & Case de terrain qui contient une maison \\
\end{tabular} 
\vspace{0.2cm} \\



\noindent \begin{tabular}{ll}
\textbf{Constante:} & RESERVATION \\
\textbf{Valeur:} & 2 \\
\textbf{Description:} & Case de terrain r�serv�e \\
\end{tabular} 
\vspace{0.2cm} \\



\noindent \begin{tabular}{ll}
\textbf{Constante:} & MONUMENT \\
\textbf{Valeur:} & 3 \\
\textbf{Description:} & Case de terrain qui contient un monument \\
\end{tabular} 
\vspace{0.2cm} \\



\noindent \begin{tabular}{ll}
\textbf{Constante:} & ROUTE \\
\textbf{Valeur:} & 4 \\
\textbf{Description:} & Case de terrain qui contient une route \\
\end{tabular} 
\vspace{0.2cm} \\

\subsubsection{Bornes, tailles}

\noindent \begin{tabular}{ll}
\textbf{Constante:} & MAX\_MONUMENTS \\
\textbf{Valeur:} & 13 \\
\textbf{Description:} & todo \\
\end{tabular} 
\vspace{0.2cm} \\



\noindent \begin{tabular}{ll}
\textbf{Constante:} & TAILLE\_CARTE \\
\textbf{Valeur:} & 100 \\
\textbf{Description:} & todo \\
\end{tabular} 
\vspace{0.2cm} \\



\subsection{Fonctions d'information}
Toutes les fonctions peuvent renvoyer les constantes
\textbf{BAD\_ARGUMENT} quand au moins un des
arguments est incorrect (m�me si cela n'est pas pr�cis� pour chaque
fonction). Par exemple, cela se produit si vous appelez
la fonction \texttt{type\_case} avec \texttt{x=13} et \texttt{y=5142}.

\begin{minipage}{\linewidth}
\functitle{type\_case}

\begin{lst-c++}
int type_case(int x, int y)
\end{lst-c++}

\noindent
\begin{tabular}[t]{@{\extracolsep{0pt}}>{\bfseries}lp{10cm}}
Description~: & Renvoie le type de la case \\


Parametres~: &
\begin{tabular}[t]{@{\extracolsep{0pt}}ll}
    
    
      
        \textsl{x}~: & todo \\
      
    
      
        \textsl{y}~: & todo \\
      
    
  \end{tabular} \\






\end{tabular} \\[0.3cm]
\end{minipage}


\begin{minipage}{\linewidth}
\functitle{valeur\_case}

\begin{lst-c++}
int valeur_case(int x, int y)
\end{lst-c++}

\noindent
\begin{tabular}[t]{@{\extracolsep{0pt}}>{\bfseries}lp{10cm}}
Description~: & Renvoie la valeur de la case \\


Parametres~: &
\begin{tabular}[t]{@{\extracolsep{0pt}}ll}
    
    
      
        \textsl{x}~: & todo \\
      
    
      
        \textsl{y}~: & todo \\
      
    
  \end{tabular} \\






\end{tabular} \\[0.3cm]
\end{minipage}


\begin{minipage}{\linewidth}
\functitle{appartenance}

\begin{lst-c++}
int appartenance(int x, int y)
\end{lst-c++}

\noindent
\begin{tabular}[t]{@{\extracolsep{0pt}}>{\bfseries}lp{10cm}}
Description~: & todo \\


Parametres~: &
\begin{tabular}[t]{@{\extracolsep{0pt}}ll}
    
    
      
        \textsl{x}~: & todo \\
      
    
      
        \textsl{y}~: & todo \\
      
    
  \end{tabular} \\






\end{tabular} \\[0.3cm]
\end{minipage}


\begin{minipage}{\linewidth}
\functitle{type\_monument}

\begin{lst-c++}
int type_monument(int x, int y)
\end{lst-c++}

\noindent
\begin{tabular}[t]{@{\extracolsep{0pt}}>{\bfseries}lp{10cm}}
Description~: & todo \\


Parametres~: &
\begin{tabular}[t]{@{\extracolsep{0pt}}ll}
    
    
      
        \textsl{x}~: & todo \\
      
    
      
        \textsl{y}~: & todo \\
      
    
  \end{tabular} \\






\end{tabular} \\[0.3cm]
\end{minipage}


\begin{minipage}{\linewidth}
\functitle{portee\_monument}

\begin{lst-c++}
int portee_monument(int num_monument)
\end{lst-c++}

\noindent
\begin{tabular}[t]{@{\extracolsep{0pt}}>{\bfseries}lp{10cm}}
Description~: & todo \\


Parametres~: &
\begin{tabular}[t]{@{\extracolsep{0pt}}ll}
    
    
      
        \textsl{num\_monument}~: & todo \\
      
    
  \end{tabular} \\






\end{tabular} \\[0.3cm]
\end{minipage}


\begin{minipage}{\linewidth}
\functitle{prestige\_monument}

\begin{lst-c++}
int prestige_monument(int num_monument)
\end{lst-c++}

\noindent
\begin{tabular}[t]{@{\extracolsep{0pt}}>{\bfseries}lp{10cm}}
Description~: & todo \\


Parametres~: &
\begin{tabular}[t]{@{\extracolsep{0pt}}ll}
    
    
      
        \textsl{num\_monument}~: & todo \\
      
    
  \end{tabular} \\






\end{tabular} \\[0.3cm]
\end{minipage}


\begin{minipage}{\linewidth}
\functitle{numero\_tour}

\begin{lst-c++}
int numero_tour()
\end{lst-c++}

\noindent
\begin{tabular}[t]{@{\extracolsep{0pt}}>{\bfseries}lp{10cm}}
Description~: & Renvoie le numero du tour \\







\end{tabular} \\[0.3cm]
\end{minipage}


\begin{minipage}{\linewidth}
\functitle{commence}

\begin{lst-c++}
int commence()
\end{lst-c++}

\noindent
\begin{tabular}[t]{@{\extracolsep{0pt}}>{\bfseries}lp{10cm}}
Description~: & Numero du joueur qui commence \\







\end{tabular} \\[0.3cm]
\end{minipage}


\begin{minipage}{\linewidth}
\functitle{montant\_encheres}

\begin{lst-c++}
int montant_encheres(int num_joueur)
\end{lst-c++}

\noindent
\begin{tabular}[t]{@{\extracolsep{0pt}}>{\bfseries}lp{10cm}}
Description~: & todo \\


Parametres~: &
\begin{tabular}[t]{@{\extracolsep{0pt}}ll}
    
    
      
        \textsl{num\_joueur}~: & todo \\
      
    
  \end{tabular} \\






\end{tabular} \\[0.3cm]
\end{minipage}


\begin{minipage}{\linewidth}
\functitle{score}

\begin{lst-c++}
int score(int num_joueur)
\end{lst-c++}

\noindent
\begin{tabular}[t]{@{\extracolsep{0pt}}>{\bfseries}lp{10cm}}
Description~: & todo \\


Parametres~: &
\begin{tabular}[t]{@{\extracolsep{0pt}}ll}
    
    
      
        \textsl{num\_joueur}~: & id du joueur \\
      
    
  \end{tabular} \\






\end{tabular} \\[0.3cm]
\end{minipage}


\begin{minipage}{\linewidth}
\functitle{finances}

\begin{lst-c++}
int finances(int num_joueur)
\end{lst-c++}

\noindent
\begin{tabular}[t]{@{\extracolsep{0pt}}>{\bfseries}lp{10cm}}
Description~: & todo \\


Parametres~: &
\begin{tabular}[t]{@{\extracolsep{0pt}}ll}
    
    
      
        \textsl{num\_joueur}~: & id du joueur \\
      
    
  \end{tabular} \\






\end{tabular} \\[0.3cm]
\end{minipage}


\begin{minipage}{\linewidth}
\functitle{monument\_en\_cours}

\begin{lst-c++}
int monument_en_cours()
\end{lst-c++}

\noindent
\begin{tabular}[t]{@{\extracolsep{0pt}}>{\bfseries}lp{10cm}}
Description~: & todo \\







\end{tabular} \\[0.3cm]
\end{minipage}


\begin{minipage}{\linewidth}
\functitle{distance}

\begin{lst-c++}
int distance(int x1, int y1, int x2, int y2)
\end{lst-c++}

\noindent
\begin{tabular}[t]{@{\extracolsep{0pt}}>{\bfseries}lp{10cm}}
Description~: & todo \\


Parametres~: &
\begin{tabular}[t]{@{\extracolsep{0pt}}ll}
    
    
      
        \textsl{x1}~: & la colonne du premier point \\
      
    
      
        \textsl{y1}~: & la ligne du premier point \\
      
    
      
        \textsl{x2}~: & la colonne du second point \\
      
    
      
        \textsl{y2}~: & la ligne du second point \\
      
    
  \end{tabular} \\






\end{tabular} \\[0.3cm]
\end{minipage}


\begin{minipage}{\linewidth}
\functitle{route\_possible}

\begin{lst-c++}
int route_possible(int x, int y)
\end{lst-c++}

\noindent
\begin{tabular}[t]{@{\extracolsep{0pt}}>{\bfseries}lp{10cm}}
Description~: & todo \\


Parametres~: &
\begin{tabular}[t]{@{\extracolsep{0pt}}ll}
    
    
      
        \textsl{x}~: & la colonne \\
      
    
      
        \textsl{y}~: & la ligne \\
      
    
  \end{tabular} \\






\end{tabular} \\[0.3cm]
\end{minipage}


\begin{minipage}{\linewidth}
\functitle{construction\_possible}

\begin{lst-c++}
int construction_possible(int x, int y)
\end{lst-c++}

\noindent
\begin{tabular}[t]{@{\extracolsep{0pt}}>{\bfseries}lp{10cm}}
Description~: & todo \\


Parametres~: &
\begin{tabular}[t]{@{\extracolsep{0pt}}ll}
    
    
      
        \textsl{x}~: & la colonne \\
      
    
      
        \textsl{y}~: & la ligne \\
      
    
  \end{tabular} \\






\end{tabular} \\[0.3cm]
\end{minipage}


\begin{minipage}{\linewidth}
\functitle{cout\_achat\_maison}

\begin{lst-c++}
int cout_achat_maison()
\end{lst-c++}

\noindent
\begin{tabular}[t]{@{\extracolsep{0pt}}>{\bfseries}lp{10cm}}
Description~: & todo \\







\end{tabular} \\[0.3cm]
\end{minipage}


\begin{minipage}{\linewidth}
\functitle{cout\_achat\_route}

\begin{lst-c++}
int cout_achat_route()
\end{lst-c++}

\noindent
\begin{tabular}[t]{@{\extracolsep{0pt}}>{\bfseries}lp{10cm}}
Description~: & todo \\







\end{tabular} \\[0.3cm]
\end{minipage}


\begin{minipage}{\linewidth}
\functitle{mon\_tour}

\begin{lst-c++}
int mon_tour()
\end{lst-c++}

\noindent
\begin{tabular}[t]{@{\extracolsep{0pt}}>{\bfseries}lp{10cm}}
Description~: & todo \\







\end{tabular} \\[0.3cm]
\end{minipage}

\subsection{Fonctions d'action}
Toutes les fonctions peuvent renvoyer les constantes
\textbf{BAD\_ARGUMENT} quand au moins un des
arguments est incorrect (m�me si cela n'est pas pr�cis� pour chaque
fonction), ou \textbf{TOO\_MUCH\_ORDERS} quand vous avez envoy� trop 
d'ordres � l'un de vos hamsters. TODO

\begin{minipage}{\linewidth}
\functitle{construire\_route}

\begin{lst-c++}
int construire_route(int x, int y)
\end{lst-c++}

\noindent
\begin{tabular}[t]{@{\extracolsep{0pt}}>{\bfseries}lp{10cm}}
Description~: & todo \\


Parametres~: &
\begin{tabular}[t]{@{\extracolsep{0pt}}ll}
    
    
      
        \textsl{x}~: & la colonne \\
      
    
      
        \textsl{y}~: & la ligne \\
      
    
  \end{tabular} \\






\end{tabular} \\[0.3cm]
\end{minipage}


\begin{minipage}{\linewidth}
\functitle{construire\_maison}

\begin{lst-c++}
int construire_maison(int x, int y)
\end{lst-c++}

\noindent
\begin{tabular}[t]{@{\extracolsep{0pt}}>{\bfseries}lp{10cm}}
Description~: & todo \\


Parametres~: &
\begin{tabular}[t]{@{\extracolsep{0pt}}ll}
    
    
      
        \textsl{x}~: & la colonne \\
      
    
      
        \textsl{y}~: & la ligne \\
      
    
  \end{tabular} \\






\end{tabular} \\[0.3cm]
\end{minipage}


\begin{minipage}{\linewidth}
\functitle{reserver\_case}

\begin{lst-c++}
int reserver_case(int x, int y)
\end{lst-c++}

\noindent
\begin{tabular}[t]{@{\extracolsep{0pt}}>{\bfseries}lp{10cm}}
Description~: & todo \\


Parametres~: &
\begin{tabular}[t]{@{\extracolsep{0pt}}ll}
    
    
      
        \textsl{x}~: & la colonne \\
      
    
      
        \textsl{y}~: & la ligne \\
      
    
  \end{tabular} \\






\end{tabular} \\[0.3cm]
\end{minipage}


\begin{minipage}{\linewidth}
\functitle{detruire\_maison}

\begin{lst-c++}
int detruire_maison(int x, int y)
\end{lst-c++}

\noindent
\begin{tabular}[t]{@{\extracolsep{0pt}}>{\bfseries}lp{10cm}}
Description~: & todo \\


Parametres~: &
\begin{tabular}[t]{@{\extracolsep{0pt}}ll}
    
    
      
        \textsl{x}~: & la colonne \\
      
    
      
        \textsl{y}~: & la ligne \\
      
    
  \end{tabular} \\






\end{tabular} \\[0.3cm]
\end{minipage}


\begin{minipage}{\linewidth}
\functitle{vendre\_maison}

\begin{lst-c++}
int vendre_maison(int x, int y)
\end{lst-c++}

\noindent
\begin{tabular}[t]{@{\extracolsep{0pt}}>{\bfseries}lp{10cm}}
Description~: & todo \\


Parametres~: &
\begin{tabular}[t]{@{\extracolsep{0pt}}ll}
    
    
      
        \textsl{x}~: & la colonne \\
      
    
      
        \textsl{y}~: & la ligne \\
      
    
  \end{tabular} \\






\end{tabular} \\[0.3cm]
\end{minipage}


\begin{minipage}{\linewidth}
\functitle{encherir}

\begin{lst-c++}
int encherir(int montant)
\end{lst-c++}

\noindent
\begin{tabular}[t]{@{\extracolsep{0pt}}>{\bfseries}lp{10cm}}
Description~: & todo \\


Parametres~: &
\begin{tabular}[t]{@{\extracolsep{0pt}}ll}
    
    
      
        \textsl{montant}~: & Le montant de votre ench�re \\
      
    
  \end{tabular} \\






\end{tabular} \\[0.3cm]
\end{minipage}


\begin{minipage}{\linewidth}
\functitle{construire\_monument}

\begin{lst-c++}
int construire_monument(int x, int y)
\end{lst-c++}

\noindent
\begin{tabular}[t]{@{\extracolsep{0pt}}>{\bfseries}lp{10cm}}
Description~: & todo \\


Parametres~: &
\begin{tabular}[t]{@{\extracolsep{0pt}}ll}
    
    
      
        \textsl{x}~: & la colonne \\
      
    
      
        \textsl{y}~: & la ligne \\
      
    
  \end{tabular} \\






\end{tabular} \\[0.3cm]
\end{minipage}



\end{document}



\newpage
\section{Et que le meilleur gagne} %----------------------------------

Il vous reste désormais 36h pour être le plus performant, mais c'est
pour la bonne cause !

\end{document}

% ====================================================================
% Zavie - Julien Guertault
% LLB - Laurent Le Brun
% Yabo - Maxime Van Noppen
