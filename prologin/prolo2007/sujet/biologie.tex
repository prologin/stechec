
\section{Petit cours de biologie} %-----------------------------------

Vous l'avez compris, ce qui fait le succès d'une fiction médicale,
c'est la capacité d'assomer le spectateur avec un vocabulaire auquel
il ne comprend absolument rien, mais rigoureusement correct. Ici nous
ne dérogeons pas à la règle, et afin de vous permettre de parler une
nouvelle langue, nous vous proposons ici un petit rappel de biologie
tiré d'articles de Wikipédia\footnote{Pour le cas fort dommage où vous
ne connaîtriez pas~: http://www.wikipedia.org}. Avant même de vous en
rendre compte, vous verrez que vous utiliserez tout ce vocabulaire
abscons avec autant d'aisance que vous le faites déjà en informatique.

\subsection{Les leucocytes}

Les leucocytes sont ce que l'on appelle couramment les globules
blancs. Ils sont fabriqués par la moelle osseuse, passent leur temps à
se ballader partout dans le corps, et ont un rôle défensif. Ils
produisent des anticorps, vérifient la santé des cellules et détectent
les cellules infectées, identifient les corps étrangers (bactéries,
virus), et enfin font le ménage.\\

Il existe plusieurs types de leucocytes, avec chacun leur rôle, mais
dans le sujet nous ne faisons cependant pas cette distinction.

\subsection{Les anticorps}

Les anticorps sont des protéines en forme de Y, qui sont produites par
les leucocytes, et qui peuvent se fixer sur certaines toxines. En se
fixant sur les mollécules de fixation d'un virus par exemple, ils
l'empêchent lui de se fixer sur une cellule, le neutralisant.

\subsection{Les bactéries}

Les bactéries sont des organismes unicellulaires de tailles et
capacitées très variées. Lorsque les conditions sont optimales, le
nombres d'individus d'une colonie de bactéries peut doubler toutes les
vingt minutes. Aussi une colonnie à tendance à avoir une croissance
exponentielle au début, puis linéaire lorsque les plus vieilles
meurent, et enfin décroissante lorsque les ressources sont épuisées.
Au sein du corps humain elles sont dix fois plus nombreuses que les
cellules.

\subsection{Les virus}

Les virus sont des corps, généralement de petite taille, utilisant des
cellules hôtes pour se reproduire. Pour simplifier, l'infection d'une
cellule par un virus se passe de la façon suivante~: le virus se fixe
à la cellule hôte, fait fusionner son gène avec celui de la cellule,
qui se met alors à produire des virus au détriment de sa propre
membrane.
