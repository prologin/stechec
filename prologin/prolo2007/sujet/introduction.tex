\section{Faire de vous des héros}

Cette année, ce sont les quinze ans de Prologin. Déjà. Vous vous
rendez compte ? Alors forcément pour un tel anniversaire, on veut
faire quelque chose de bien, quelque chose qui marque les esprits.\\

Chaque année, afin que chacun puisse montrer fièrement à ses parents
et amis à quoi ressemble le prestigieux concours auxquels ils
participent ``Tiens regarde, c'est moi juste là, à côté de la machine
à barbe à papa~!'', nous proposons une vidéo de l'évènement
peu\footnote{Bon parfois tout est relatif. ;-)} de temps après. C'est
pour cela que vous verrez se ballader avec une caméra des gens très
intéressés par ce que vous faites et qui, bizarrement, ne taperont pas
une ligne de code. Ne soyez donc pas inquiets, ce ne sont pas des
espions envoyés par d'autres candidats.\\

Cette année nous pensions tirer parti de cette vidéo pour faire de
l'audience, de l'audimat, faire de vous des stars, et être remarqués
par tous les grands producteurs. Pour cela, rien de tel que de
l'action, du sexe, de l'humour, du suspens, et de
l'émotion\footnote{Notez qu'en ne gardant que les deux premiers, ça
fait un truc parfaitement moisi mais financièrement très
rentable.}. Malheureusement, ici nous faisons de l'informatique, et il
faut bien reconnaître que l'action y est tout de même relativement
discrète\footnote{Mais heureusement en dehors de l'informatique, il y
a justement de l'action pendant une finale Prologin~!}, le sexe limité
à de la drague sur IRC (or vous êtes désormais coupés du monde), et
l'humour d'informaticien passe quant à lui assez mal malheureusement~:
``Ta mère const typée en string~!'' \textsl{Rires enregistrés}.\\

Enfin les deux derniers éléments sont difficile à faire communiquer.
Forcément, on imagine la difficulté que peut représenter pour un
metteur en scène le défi de faire ressortir de façon palpable toute la
tension liée à une compilation de code template en C++~: cinq minutes
de silence devant un écran qui ne dit rien, et tout à coup mille
lignes d'erreur, ou juste rien, voire un timide ``Compilation
terminée''. De même il semble peu aisé de faire comprendre au
spectateur le plaisir carnassier d'avoir trouvé l'origine d'une erreur
de segmentation, et la traque l'ayant précédée, à coups d'affichages
de pile en exécutant du code pas à pas, n'est pas simple à rendre
captivante non plus~: ``Ce pointeur nul, j'en fait une affaire
personnelle~!''. Alors les rares réalisateurs qui s'aventurent sur le
terrain ont recours à des procédés douteux, comme par exemple
représenter l'administrateur système comme un délinquant se déplaçant
dans ses salles serveur en skateboard et codant des virus avec des
interfaces en 3D.\\

Pourtant il existe des domaines auxquels les gens ne comprennent
absolument rien non plus et qui connaissent pourtant de gros succès
populaires. Prenez par exemple la médecine~: tournez une hypothétique
série télévisée dans laquelle les héros seraient de brillants médecins
d'un hôpital de Chicago et lanceraient des ``Je veux
NFS\footnote{Numération formule sanguine, aussi appelé
\emph{hémogramme}, ou \emph{examen hématologique complet}, ou plus
simplement \emph{hémato}~: vous l'avez déjà entendu n'est ce pas~?},
iono\footnot{Ionogramme~: concentration sanguine des différents
ions.}~!'' et autres ``Il bradycardise\footnot{La bradycardie se
caractérise par un rythme cardiaque trop bas par rapport à la
normale.}~!'' comme on crie des attaques dans un manga, et vous faites
de vos acteurs des stars que l'on s'arrache.\\

Devant ce constat, nous sommes arrivés à la conclusion qu'il fallait
quelque chose de nouveau pour pimenter la finale. C'est la raison pour
laquelle cette année, nous introduisons de la médecine~! Cette
décision étant prise, vient donc le choix du domaine. La médecine
urgentiste a un côté action qui lui réussi beaucoup, mais nous n'avons
pas les moyens de ce genre de production, et avoir cent personnes
réclamant avec conviction des scalpels, compresses, et défibrilateurs
dans un bloc opératoire risquerait de donner un résultat un peu trop
chaotique. Le sujet est de toute façon déjà cliché et il vaudrait donc
mieux innover. Il faut donc se placer dans un cadre ou plutôt à une
autre échelle.\\


\section{De l'infiniment grand à l'infiniment petit}

Vous avez probablement déjà vu cette vidéo stupéfiante dans laquelle
la caméra, partant d'un couple allongé dans l'herbe dans un parc,
s'éloigne à vitesse exponentielle, laissant voir la ville, le pays, la
planète, le système solaire, la galaxie, pour arriver aux confins de
l'univers~?  La caméra revient ensuite, puis commence à descendre au
niveau de la main, à l'échelle de l'insecte, de la cellule, de la
molécule, de l'atome. Cette année nous allons nous placer à l'échelle
de la cellule.\\

Saviez-vous que l'année dernière une équipe de chercheurs simulait
quelques nanosecondes de la vie d'un virus dans de l'eau, à l'échelle
atomique~? L'expérience devait être extrêmement intéressante.
Malheureusement n'espérez pas pouvoir faire de même chez vous avant
plusieurs années~: le temps que calculer les interactions entre un
million d'atomes soit envisageable. Toutefois, tout en restant bien
loin de ce niveau de détail dans le modèle utilisé, nous vous
proposons aujourd'hui une expérience du même registre.\\

Savez-vous de quoi est fait votre système immunitaire~? Savez vous ce
que sont un virus, une bactérie, un anticorps, un globule blanc, ou
tout simplement une cellule~? Avez-vous idée de ce qu'il se passe à
cette échelle lors d'une infection~? Si vous aviez le contrôle de ces
globules blancs, que feriez-vous~? Si vous étiez un globule blanc, que
feriez vous~?\\

Voici ce à quoi nous vous proposons de réfléchir aujourd'hui.
