%LaTeX Document

%\documentclass[a4paper,twoside,12pt]{book}
\documentclass[a4paper,twoside,12pt]{article}

\usepackage[latin1]{inputenc}
\usepackage[T1]{fontenc}
\usepackage[french]{babel}
\usepackage{a4}
\usepackage{geometry}
\usepackage{graphicx}
\usepackage{multicol}
\usepackage{fancyhdr}
\usepackage{palatino}
\usepackage{subfigure}
\usepackage{verbatim,fancyvrb}
\renewcommand\ttdefault{cmtt}

%\NoAutoSpaceBeforeFDP
\geometry{bindingoffset=5mm,hmarginratio=1:1,heightrounded,headheight=15pt}

\makeindex

\begin{document}
\pagestyle{empty}
\sloppy

%\lhead[\textsl{Prologin 2005}]{\nouppercase \leftmark}
%\rhead[Sujet de la finale]{}
\lhead[\thepage]{\nouppercase \leftmark}
\rhead[\textsl{Prologin 2007} --- Sujet de la finale]{\thepage}
\cfoot{}

% Couverture =========================================================
\begin{titlepage}
\begin{center}
\includegraphics[width=\linewidth]{logo}\\
\vspace{4cm}
\Huge
\textbf{Il \'etait une fois la vie...}\\
\vspace{2cm}
\normalsize
Sujet de la finale du Concours National d'Informatique\\
DATE\\
\end{center}
\end{titlepage}

% Sommaire ===========================================================
\cleardoublepage
\tableofcontents

% Corps ==============================================================
\cleardoublepage
\setcounter{page}{1}
\pagestyle{fancy}
\parskip=6pt plus 3pt

\section{Il \'etait une fois\ldots} %-----------------------------------

 \subsection{\ldots{} la vie}

TODO: Intro.

  \subsection{Le champ de bataille}

  \subsection{Globule blanc}

    \subsubsection{Description}

    FIXME: Mettre une image de globule blanc version Il était une fois la vie.

    Le globule blanc\footnote{leucocyte de son petit nom} est l'unité maîtresse
    des défenses immunitaires. Vous allez donc vous retrouver, Général malgré
    vous, à la tête d'une unité de ces valeureux soldats prêt à tout donner,
    même leur vie, pour protéger les cellules. Nous n'insisterons jamais assez
    sur ce dernier point : votre raison d'être est de protéger les cellules.

    FIXME: Paragraphe sur le QG des GB.

    FIXME: Nombre de points à partager.

    Lorsque vous déclenchez le processus de genèse d'un globule blanc, vous
    pouvez faire varier sa composition afin d'influer sur ses performances, et
    donc lui attribuer un rôle spécifique au sein de votre unité. Ajoutez des
    protéines et votre globule blanc sera à même de lancer plus d'anticorps,
    plus de lipides et c'est sa vitesse de phagocytose qui s'accroît, plus
    d'oxygène lui permettra de communiquer plus abondamment avec ses compagnons
    d'armes et plus de fer lui permettra de voir sa capacité de détection
    des intrus à longue distance s'accroître.

    Attention cependant à ne pas le surcharger de composants, vous risqueriez de
    vous retrouver avec un globule dégénéré qui serait totalement inefficace !

    Pour résumer, vous avez <FIXME>N</FIXME> points de compétences à répartir
    parmi les quatre caractéristiques suivantes :

    \begin{itemize}
      \item Nombre d'anticorps lâchés
      \item Vitesse de phagocytose
      \item Nombre de messages à envoyer
      \item Champ de vision
    \end{itemize}

    \subsubsection{Les anticorps \ldots}

    FIXME: Image anticorps.

    Les anticorps sont non pas de sortes de petits robots mais des protéines
    que votre globule blanc peut libérer en plus ou moins grande quantité et
    qui vont attaquer les virus et bactéries. Quand vous décidez de libérer
    vos anticorps, ils sont tout d'abord tous sur votre case et puis tour
    après tour vont se répandre autour du vous, essentiellement vers le bas
    dans le sens de l'écoulement du sang.

      \subsubsection*{\ldots vs les virus}

      Pour que vos anticorps soient efficaces contre un type de virus il faut
      \textbf{absolument} que le globule blanc qui les libère ait déjà tué au
      moins un virus de ce type par phagocytose.

      Si les anticorps sont efficaces contre le virus rencontré ils vont se
      fixer sur lui pour le tuer. Chaque anticorps qui se fixe sur un virus
      pénètre à l'intérieur de ce dernier avant de se dissoudre en une toxine
      mortelle pour le virus. Un anticorps ne suffit pas pour tuer un virus et
      suivant la résistance de ce dernier il en faudra plus ou moins. Si la
      quantité d'anticorps qui se sont fixés sur le virus n'est pas suffisante
      pour le tuer, il survivra à l'attaque mais en ressortira notablement
      affaibli.

      \subsubsection*{\ldots vs les bactéries}

      Lorsque des anticorps arrivent sur une case où des bactéries sont
      présentes ils attaquent la colonie. Chaque anticorps va attaquer une
      bactérie, fusionner avec elle et la tuer. Ceci implique qu'un anticorps
      ne peut tuer qu'une bactérie et puis disparaît.

    \subsubsection{La phagocytose}

    FIXME: Image d'un GB qui bouffe un truc.

    La phagocytose est un autre moyen que vos globules blanc ont de repousser
    l'invasion. Métaphoriquement, ils vont \emph{manger} le virus ou la
    bactérie cible. En fait il va l'entourer de lipided et l'intégrer à son
    propre organisme. Il est important de noter qu'une fois qu'un globule
    blanc à consommé un virus d'un certain type il est capable de produire des
    anticorps pour lutter contre ce type de virus.
      
    \subsubsection{Les messages}

    lol

    \subsubsection{Le champ de vision}

    lol

\end{document}
