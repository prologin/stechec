

%
% Ce fichier a ete genere avec gen/make_tex.rtex
% Ne faites pas l'autiste, ne le modifiez pas directement 
%

\subsection{Constantes}

\noindent \begin{tabular}{lp{15cm}}
\textbf{Constante:} & TAILLE\_TERRAIN \\
\textbf{Valeur:} & 30 \\
\textbf{Description:} & Taille du terrain \\
\end{tabular} 
\vspace{0.2cm} \\



\noindent \begin{tabular}{lp{15cm}}
\textbf{Constante:} & FIN\_PARTIE \\
\textbf{Valeur:} & 150 \\
\textbf{Description:} & Nombre de tours par partie \\
\end{tabular} 
\vspace{0.2cm} \\



\noindent \begin{tabular}{lp{15cm}}
\textbf{Constante:} & MAX\_PA \\
\textbf{Valeur:} & 3 \\
\textbf{Description:} & Nombre de points d'action par tour \\
\end{tabular} 
\vspace{0.2cm} \\



\noindent \begin{tabular}{lp{15cm}}
\textbf{Constante:} & INTENSITE\_TRAINEE \\
\textbf{Valeur:} & 120 \\
\textbf{Description:} & Taille des traînées de moto \\
\end{tabular} 
\vspace{0.2cm} \\



\noindent \begin{tabular}{lp{15cm}}
\textbf{Constante:} & MAX\_ALLONGEMENT \\
\textbf{Valeur:} & 5 \\
\textbf{Description:} & Longueur maximale de l'allongement \\
\end{tabular} 
\vspace{0.2cm} \\



\noindent \begin{tabular}{lp{15cm}}
\textbf{Constante:} & AJOUT\_PA \\
\textbf{Valeur:} & 5 \\
\textbf{Description:} & Nombre de points d'action à rajouter avec bonus \\
\end{tabular} 
\vspace{0.2cm} \\



\subsection{Énumérations}

\functitle{erreur} \\
\noindent
\begin{tabular}[t]{@{\extracolsep{0pt}}>{\bfseries}lp{10cm}}
Description~: & Énumération représentant une erreur renvoyée par une des fonctions d'action \\
Valeurs~: &
\small
\begin{tabular}[t]{@{\extracolsep{0pt}}lp{7cm}}
    
        \textsl{OK}~: & aucune erreur n'est survenue \\
    
        \textsl{ID\_INVALIDE}~: & identifiant invalide \\
    
        \textsl{POSITION\_INVALIDE}~: & la position spécifiée est invalide \\
    
        \textsl{PLUS\_DE\_PA}~: & vous n'avez pas assez de points d'action \\
    
        \textsl{BONUS\_INVALIDE}~: & vous n'avez pas ce bonus \\
    
        \textsl{PAS\_A\_TOI}~: & l'unité n'est pas a vous \\
    
        \textsl{INTENSITE\_INVALIDE}~: & cette intensité est invalide \\
    
\end{tabular} \\
\end{tabular}



\functitle{type\_case} \\
\noindent
\begin{tabular}[t]{@{\extracolsep{0pt}}>{\bfseries}lp{10cm}}
Description~: & Énumération représentant les différents types de case \\
Valeurs~: &
\small
\begin{tabular}[t]{@{\extracolsep{0pt}}lp{7cm}}
    
        \textsl{VIDE}~: & rien n'est présent sur la case \\
    
        \textsl{OBSTACLE}~: & cette case est inaccessible \\
    
        \textsl{POINT\_CROISEMENT}~: & point de croisement de traînées \\
    
        \textsl{UNITE}~: & unité d'énergie \\
    
\end{tabular} \\
\end{tabular}



\functitle{type\_bonus} \\
\noindent
\begin{tabular}[t]{@{\extracolsep{0pt}}>{\bfseries}lp{10cm}}
Description~: & Énumération représentant les différents types de bonii \\
Valeurs~: &
\small
\begin{tabular}[t]{@{\extracolsep{0pt}}lp{7cm}}
    
        \textsl{PAS\_BONUS}~: & ceci n'est pas un bonus :-) \\
    
        \textsl{BONUS\_CROISEMENT}~: & bonus permettant de croiser deux traînées de moto sur une case \\
    
        \textsl{PLUS\_LONG}~: & bonus permettant d'agrandir une traînée de moto \\
    
        \textsl{PLUS\_PA}~: & bonus permettant d'avoir plus de points d'action \\
    
        \textsl{BONUS\_REGENERATION}~: & bonus permettant de regenerer une unité d'énergie \\
    
\end{tabular} \\
\end{tabular}



\subsection{Structures}

\functitle{position}

\begin{lst-c++}
struct position {
    int x;
    int y;
};
\end{lst-c++}

\noindent
\begin{tabular}[t]{@{\extracolsep{0pt}}>{\bfseries}lp{10cm}}
Description~: & Représente une position sur le terrain du jeu \\
Champs~: &
\small
\begin{tabular}[t]{@{\extracolsep{0pt}}lp{7cm}}
    
        \textsl{x}~: & coordonnée en X \\
    
        \textsl{y}~: & coordonnée en Y \\
    
\end{tabular} \\
\end{tabular}



\functitle{unite\_energie}

\begin{lst-c++}
struct unite_energie {
    int id;
    position pos;
    int valeur;
    int valeur_max;
};
\end{lst-c++}

\noindent
\begin{tabular}[t]{@{\extracolsep{0pt}}>{\bfseries}lp{10cm}}
Description~: & Caracteristiques d'une unité d'énergie \\
Champs~: &
\small
\begin{tabular}[t]{@{\extracolsep{0pt}}lp{7cm}}
    
        \textsl{id}~: & identifiant de l'unité d'énergie \\
    
        \textsl{pos}~: & position de l'unité d'énergie \\
    
        \textsl{valeur}~: & coefficient représentant les points d'énergie que l'unité va vous apporter \\
    
        \textsl{valeur\_max}~: & coefficient représentant la capacité de l'unité lorsqu'elle est chargée au maximum \\
    
\end{tabular} \\
\end{tabular}



\functitle{trainee\_moto}

\begin{lst-c++}
struct trainee_moto {
    int id;
    position array emplacement;
    int team;
    int intensite;
};
\end{lst-c++}

\noindent
\begin{tabular}[t]{@{\extracolsep{0pt}}>{\bfseries}lp{10cm}}
Description~: & Représente une traînée de moto sur le terrain \\
Champs~: &
\small
\begin{tabular}[t]{@{\extracolsep{0pt}}lp{7cm}}
    
        \textsl{id}~: & identifiant de la traînee \\
    
        \textsl{emplacement}~: & position de chaque composant de la traînée de moto \\
    
        \textsl{team}~: & identifiant de l'équipe qui possède cette traînée de moto \\
    
        \textsl{intensite}~: & taille maximale de la traînée \\
    
\end{tabular} \\
\end{tabular}



\subsection{Fonctions d'information}

\begin{minipage}{\linewidth}
\functitle{mon\_equipe}

\begin{lst-c++}
int mon_equipe()
\end{lst-c++}

\noindent
\begin{tabular}[t]{@{\extracolsep{0pt}}>{\bfseries}lp{10cm}}
Description~: & Retourne le numéro de votre équipe \\







\end{tabular} \\[0.3cm]
\end{minipage}


\begin{minipage}{\linewidth}
\functitle{scores}

\begin{lst-c++}
int array scores()
\end{lst-c++}

\noindent
\begin{tabular}[t]{@{\extracolsep{0pt}}>{\bfseries}lp{10cm}}
Description~: & Retourne les scores de chaque équipe \\







\end{tabular} \\[0.3cm]
\end{minipage}


\begin{minipage}{\linewidth}
\functitle{nombre\_equipes}

\begin{lst-c++}
int nombre_equipes()
\end{lst-c++}

\noindent
\begin{tabular}[t]{@{\extracolsep{0pt}}>{\bfseries}lp{10cm}}
Description~: & Retourne le nombre d'équipes sur le terrain \\







\end{tabular} \\[0.3cm]
\end{minipage}


\begin{minipage}{\linewidth}
\functitle{tour\_actuel}

\begin{lst-c++}
int tour_actuel()
\end{lst-c++}

\noindent
\begin{tabular}[t]{@{\extracolsep{0pt}}>{\bfseries}lp{10cm}}
Description~: & Retourne le numéro du tour actuel \\







\end{tabular} \\[0.3cm]
\end{minipage}


\begin{minipage}{\linewidth}
\functitle{unites\_energie}

\begin{lst-c++}
unite_energie array unites_energie()
\end{lst-c++}

\noindent
\begin{tabular}[t]{@{\extracolsep{0pt}}>{\bfseries}lp{10cm}}
Description~: & Retourne la liste des unités d'énergie \\







\end{tabular} \\[0.3cm]
\end{minipage}


\begin{minipage}{\linewidth}
\functitle{trainees\_moto}

\begin{lst-c++}
trainee_moto array trainees_moto()
\end{lst-c++}

\noindent
\begin{tabular}[t]{@{\extracolsep{0pt}}>{\bfseries}lp{10cm}}
Description~: & Retourne la liste des traînées de moto \\







\end{tabular} \\[0.3cm]
\end{minipage}


\begin{minipage}{\linewidth}
\functitle{regarder\_type\_case}

\begin{lst-c++}
type_case regarder_type_case(position pos)
\end{lst-c++}

\noindent
\begin{tabular}[t]{@{\extracolsep{0pt}}>{\bfseries}lp{10cm}}
Description~: & Retourne le type d'une case \\


Parametres~: &
\begin{tabular}[t]{@{\extracolsep{0pt}}ll}
    
    
      
        \textsl{pos}~: & position de la case \\
      
    
  \end{tabular} \\






\end{tabular} \\[0.3cm]
\end{minipage}


\begin{minipage}{\linewidth}
\functitle{regarder\_type\_bonus}

\begin{lst-c++}
type_bonus regarder_type_bonus(position pos)
\end{lst-c++}

\noindent
\begin{tabular}[t]{@{\extracolsep{0pt}}>{\bfseries}lp{10cm}}
Description~: & Retourne le type de bonus d'une case \\


Parametres~: &
\begin{tabular}[t]{@{\extracolsep{0pt}}ll}
    
    
      
        \textsl{pos}~: & position de la case \\
      
    
  \end{tabular} \\






\end{tabular} \\[0.3cm]
\end{minipage}


\begin{minipage}{\linewidth}
\functitle{regarder\_bonus}

\begin{lst-c++}
type_bonus array regarder_bonus(int equipe)
\end{lst-c++}

\noindent
\begin{tabular}[t]{@{\extracolsep{0pt}}>{\bfseries}lp{10cm}}
Description~: & Retourne la liste des bonus d'une équipe \\


Parametres~: &
\begin{tabular}[t]{@{\extracolsep{0pt}}ll}
    
    
      
        \textsl{equipe}~: & identifiant de l'équipe visée \\
      
    
  \end{tabular} \\






\end{tabular} \\[0.3cm]
\end{minipage}


\begin{minipage}{\linewidth}
\functitle{regarder\_trainee\_case}

\begin{lst-c++}
int array regarder_trainee_case(position pos)
\end{lst-c++}

\noindent
\begin{tabular}[t]{@{\extracolsep{0pt}}>{\bfseries}lp{10cm}}
Description~: & Retourne la liste des id des traînées présentes sur une case \\


Parametres~: &
\begin{tabular}[t]{@{\extracolsep{0pt}}ll}
    
    
      
        \textsl{pos}~: & position de la case \\
      
    
  \end{tabular} \\






\end{tabular} \\[0.3cm]
\end{minipage}


\begin{minipage}{\linewidth}
\functitle{case\_traversable}

\begin{lst-c++}
bool case_traversable(position pos)
\end{lst-c++}

\noindent
\begin{tabular}[t]{@{\extracolsep{0pt}}>{\bfseries}lp{10cm}}
Description~: & Retourne si une case peut être traversée par une traînée de plus \\


Parametres~: &
\begin{tabular}[t]{@{\extracolsep{0pt}}ll}
    
    
      
        \textsl{pos}~: & position de la case \\
      
    
  \end{tabular} \\






\end{tabular} \\[0.3cm]
\end{minipage}


\begin{minipage}{\linewidth}
\functitle{gain\_tour\_suivant}

\begin{lst-c++}
int gain_tour_suivant()
\end{lst-c++}

\noindent
\begin{tabular}[t]{@{\extracolsep{0pt}}>{\bfseries}lp{10cm}}
Description~: & Renvoie les points que vous allez gagner a la fin du tour \\







\end{tabular} \\[0.3cm]
\end{minipage}


\begin{minipage}{\linewidth}
\functitle{chemin}

\begin{lst-c++}
position array chemin(position p1, position p2)
\end{lst-c++}

\noindent
\begin{tabular}[t]{@{\extracolsep{0pt}}>{\bfseries}lp{10cm}}
Description~: & Renvoie le chemin le plus court entre deux points (fonction lente) \\


Parametres~: &
\begin{tabular}[t]{@{\extracolsep{0pt}}ll}
    
    
      
        \textsl{p1}~: & position de départ \\
      
    
      
        \textsl{p2}~: & position d'arrivée \\
      
    
  \end{tabular} \\






\end{tabular} \\[0.3cm]
\end{minipage}

\subsection{Fonctions d'action}

\begin{minipage}{\linewidth}
\functitle{deplacer}

\begin{lst-c++}
erreur deplacer(int id, position de, position vers)
\end{lst-c++}

\noindent
\begin{tabular}[t]{@{\extracolsep{0pt}}>{\bfseries}lp{10cm}}
Description~: & Déplace une moto \\


Parametres~: &
\begin{tabular}[t]{@{\extracolsep{0pt}}ll}
    
    
      
        \textsl{id}~: & identifiant de la moto à déplacer \\
      
    
      
        \textsl{de}~: & position de l'extrémité que l'on déplace \\
      
    
      
        \textsl{vers}~: & nouvelle position pour cette extrémité \\
      
    
  \end{tabular} \\






\end{tabular} \\[0.3cm]
\end{minipage}


\begin{minipage}{\linewidth}
\functitle{couper\_trainee\_moto}

\begin{lst-c++}
erreur couper_trainee_moto(int id, position p1, position p2, int intensite_p1)
\end{lst-c++}

\noindent
\begin{tabular}[t]{@{\extracolsep{0pt}}>{\bfseries}lp{10cm}}
Description~: & Coupe une traînée de moto en deux nouvelles traînées. « p1 » et « p2 » doivent être deux positions adjacentes occupées par une même traînée de moto. \\


Parametres~: &
\begin{tabular}[t]{@{\extracolsep{0pt}}ll}
    
    
      
        \textsl{id}~: & identifiant de la traînée de moto à couper \\
      
    
      
        \textsl{p1}~: & nouvelle extrémité de la première traînée de moto \\
      
    
      
        \textsl{p2}~: & nouvelle extrémité de la deuxième traînée de moto \\
      
    
      
        \textsl{intensite\_p1}~: & croissance restante de la moitié de la traînée de moto contenant entre] \\
      
    
  \end{tabular} \\






\end{tabular} \\[0.3cm]
\end{minipage}


\begin{minipage}{\linewidth}
\functitle{cancel}

\begin{lst-c++}
erreur cancel()
\end{lst-c++}

\noindent
\begin{tabular}[t]{@{\extracolsep{0pt}}>{\bfseries}lp{10cm}}
Description~: & Annuler l'action précédente \\







\end{tabular} \\[0.3cm]
\end{minipage}


\begin{minipage}{\linewidth}
\functitle{enrouler}

\begin{lst-c++}
erreur enrouler(int id, position p)
\end{lst-c++}

\noindent
\begin{tabular}[t]{@{\extracolsep{0pt}}>{\bfseries}lp{10cm}}
Description~: & Enrouler la traînée de moto en un point \\


Parametres~: &
\begin{tabular}[t]{@{\extracolsep{0pt}}ll}
    
    
      
        \textsl{id}~: & identifiant de la traînée de moto à enrouler \\
      
    
      
        \textsl{p}~: & point sur lequel enrouler la moto \\
      
    
  \end{tabular} \\






\end{tabular} \\[0.3cm]
\end{minipage}


\begin{minipage}{\linewidth}
\functitle{regenerer\_unite\_energie}

\begin{lst-c++}
erreur regenerer_unite_energie(int id)
\end{lst-c++}

\noindent
\begin{tabular}[t]{@{\extracolsep{0pt}}>{\bfseries}lp{10cm}}
Description~: & Régénère une unité d'énergie à son maximal \\


Parametres~: &
\begin{tabular}[t]{@{\extracolsep{0pt}}ll}
    
    
      
        \textsl{id}~: & identifiant de l'unité d'énergie à regénérer \\
      
    
  \end{tabular} \\






\end{tabular} \\[0.3cm]
\end{minipage}


\begin{minipage}{\linewidth}
\functitle{allonger\_pa}

\begin{lst-c++}
erreur allonger_pa()
\end{lst-c++}

\noindent
\begin{tabular}[t]{@{\extracolsep{0pt}}>{\bfseries}lp{10cm}}
Description~: & Allonge le tour en rajoutant des points d'action \\







\end{tabular} \\[0.3cm]
\end{minipage}


\begin{minipage}{\linewidth}
\functitle{etendre\_trainee\_moto}

\begin{lst-c++}
erreur etendre_trainee_moto(int id, int longueur)
\end{lst-c++}

\noindent
\begin{tabular}[t]{@{\extracolsep{0pt}}>{\bfseries}lp{10cm}}
Description~: & Allonge une traînée de moto. L'allongement se fera aux prochains déplacements. La longueur du prolongement doit être comprise entre 0 et MAX\_ALLONGEMENT (inclus). \\


Parametres~: &
\begin{tabular}[t]{@{\extracolsep{0pt}}ll}
    
    
      
        \textsl{id}~: & identifiant de la traînée de moto à allonger \\
      
    
      
        \textsl{longueur}~: & longueur du prolongement \\
      
    
  \end{tabular} \\






\end{tabular} \\[0.3cm]
\end{minipage}


\begin{minipage}{\linewidth}
\functitle{poser\_point\_croisement}

\begin{lst-c++}
erreur poser_point_croisement(position point)
\end{lst-c++}

\noindent
\begin{tabular}[t]{@{\extracolsep{0pt}}>{\bfseries}lp{10cm}}
Description~: & Pose un point de croisement sur une case du terrain. La case doit ne pas déjà être un point de croisement. \\


Parametres~: &
\begin{tabular}[t]{@{\extracolsep{0pt}}ll}
    
    
      
        \textsl{point}~: & position de la case sur laquelle poser le point de croisement \\
      
    
  \end{tabular} \\






\end{tabular} \\[0.3cm]
\end{minipage}


\begin{minipage}{\linewidth}
\functitle{fusionner}

\begin{lst-c++}
erreur fusionner(int id1, position pos1, int id2, position pos2)
\end{lst-c++}

\noindent
\begin{tabular}[t]{@{\extracolsep{0pt}}>{\bfseries}lp{10cm}}
Description~: & Fusionner deux traînées de moto. Les deux doivent appartenir à la même équipe, mais doivent être deux traînées distinctes. « pos1 » et « pos2 » doivent être adjacentes et occupées respectivement par « id1 » et « id2 ». \\


Parametres~: &
\begin{tabular}[t]{@{\extracolsep{0pt}}ll}
    
    
      
        \textsl{id1}~: & identifiant de la première traînée \\
      
    
      
        \textsl{pos1}~: & extrémité à fusionner de la première traînée \\
      
    
      
        \textsl{id2}~: & identifiant de la seconde traînée \\
      
    
      
        \textsl{pos2}~: & extrémité à fusionner de la seconde traînée \\
      
    
  \end{tabular} \\






\end{tabular} \\[0.3cm]
\end{minipage}
