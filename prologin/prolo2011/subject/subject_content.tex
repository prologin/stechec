\section{Introduction}
\emph{Greetings, Program! I'm Castor! Your host! Provider of any and
  all, entertainment, and diversions\ldots{}}

\emph{\ldots{}at your service.}\\

Ce monde de programmes, régi par des programmes pour les programmes,
est dirigé d'un pointeur de fer par Clu. Chargé de concevoir un monde
parfait, Clu fait régner la terreur, désactivant sans pitié tout programme
faisant un pas de travers.\\

Conçu à l'origine pour construire un monde parfait, Clu et ses équipes
sont prêts à tout pour atteindre cet objectif, quel qu'en soit le prix
à payer.

Et le monde, c'est aussi notre Terre telle que nous la connaissons.\\

Depuis qu'il en connait l'existence, Clu ne travaille plus qu'à
atteindre le monde réel, pour y porter sa vision du monde parfait.

La seule passerelle entre son monde numérique et le notre est le
Portail.\\

Unique point d'accès, le Portail ne peut être fonctionner qu'à
condition d'être alimenté par des quantités astronomiques d'énergie.

Et Clu a déjà commencé à en emmagasiner\ldots{}\\

        \subsection{Votre mission}
        \emph{In there is a new world! In there is our future! In
          there is our destiny!}

Votre mission : pénétrez sur la Grille, et empêchez à tout prix Clu
d'arriver à ses fins, en le privant des sources d'énergie.

Mais attention, Programme. Le suicide n'est pas une option. Vous
\textbf{devez} revenir, et pour cela passer le Portail.

\subsection{La Grille}
Vous vous battrez sur la Grille, environnement numérique carré, dont vous
connaissez la taille.

Cette Grille se découpe en cases, sur lesquels sont notamment
éparpillées les précieuses sources d'énergie et les points de stockage.

\newpage
\section{Les règles}
\emph{Patience, Sam Flynn. All of your questions will be answered soon.}

//FIXME histoire

        \subsection{Les Programmes}
        \emph{Attention, Program. You will receive, an identity disc. Everything you do, or learn, will be imprinted on this disc. If you, lose your disc, or fail, to follow commands, you will be subject to immediate de-resolution.}\\

A chaque tour, vous donnerez des ordres à vos Programmes. Ces derniers
sont des entités capables d'interagir avec la Grille. Ils sont
représentés par des motos.

Soyez vigilants, tout ordre vous coûtera un PA\footnote{Point
  d'Action}, et ces PA sont rares. La Grille vous en donne un certain
nombre en début de tour, mais reprendra les PA inutilisés en fin de tour.

\subsection{La Grille}
        \emph{The Grid. A digital frontier. I tried to picture clusters
          of information as they moved through the computer. What did they
          look like? Ships, motorcycles? Were the circuits like freeways? I
          kept dreaming of a world I thought I'd never see. And then, one
          day, I got in\ldots{}}\\

La Grille n'est pas qu'un simple support numérique vide. Vous pourrez
y rencontrer :

\begin{itemize}
  \item \emph{un obstacle} : les obstacles sont infranchissables,
    et doivent être contournés.
    %\vspace{1cm}
    \begin{figure}[!h]
    \centering
    \includegraphics{../data/graphics/terrain-obstacle.png}
    \caption{Un obstacle}
    \end{figure}

  \item \emph{un bonus} : vous pouvez ramasser les bonus présents
    sur la Grille en vous déplaçant dessus. Les différents bonus et
    leurs utilisations sont détaillées dans la section sur les
    bonus, page \pageref{section-bonus}.
  \item \emph{un point de croisement} : sur ces cases, il est
    possible pour (au moins) deux traînées de se croiser sans dommage.
%\vspace{1cm}
    \begin{figure}[!h]
    \centering
    \includegraphics{../data/graphics/terrain-point_croisement.png}
    \caption{Un point de croisement}
    \end{figure}
  \item \emph{une source} : ces cases contiennent soit une
    source d'énergie, soit un point de stockage.
\end{itemize}

Sans oublier bien entendu votre adversaire, ses Programmes et surtout
ses traces.\\

\subsection{Gestion de l'énergie}
Vous êtes-vous déjà demandé pourquoi les traînées des motos sont
dangeureuses ? C'est parce qu'elles transportent de l'énergie !\\

Vous l'aurez compris, votre objectif est de transférer un maximum
d'énergie des sources vers les points de stockage, et d'empêcher votre
adversaire de le faire.

Pour cela, il suffit de connecter une source à un point de stockage,
grâce à vos motos. Une trainée reliant une source à un point de
stockage est dite active.\\

\subsubsection{I got the power!}

Chaque source ou point de stockage est caractérisé par une capacité
maximale d'énergie. Une source délivre 1 unité d'énergie par tour à
chaque trainée active. De la même façon, un point de stockage accepte
1u par tour.

La capacité réelle d'une source correspond donc à la quantité
d'énergie encore disponible pour le transport ; alors que la capacité
réelle d'un point de stockage correspond à la quantité d'énergie
pouvant encore être emmagasinée.

Lorsque la capacité réelle atteint 0u, la source (respectivement le
point de stockage) est dit épuisé.\\

\begin{figure}[!h]
\centering
\begin{tabular}{c|c}
    \includegraphics{../data/graphics/source_energie-on-producteur.png}
%    \caption{Source active}

 &

    \includegraphics{../data/graphics/source_energie-off-producteur.png}
%    \caption{Source épuisée}
\\
        Source active & Source épuisée \\ \hline

    \includegraphics{../data/graphics/source_energie-on-consommateur.png}
%    \caption{Point de stockage actif}

&

    \includegraphics{../data/graphics/source_energie-off-consommateur.png}
%    \caption{Point de stockage épuisé}
\\
        Point de stockage actif & Point de stockage épuisé

\end{tabular}
\end{figure}

\subsubsection{Rechargement}

Une fois épuisées, les sources d'énergie reconstituent lentement leurs
stocks, au rythme d'une unité par tour.



        \subsection{Interagir avec la Grille}
        \emph{This is Blue Leader to Blue Bikes. Run these guys into your jet walls.}

        //FIXME les actions


\newpage
\section{Les bonus} \label{section-bonus}
\emph{Now for some real user power.}

//FIXME

\newpage
\section{Time's up!}
\emph{End of Line, man.}

//FIXME fin de partie

\newpage
\section{Conclusion}
